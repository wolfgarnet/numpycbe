In this master thesis, we will explore the capabilities of
the \CBE{}, a heterogeneous single chip multi processor. Especially
how capable it is, when doing scientific computing using Python.\\

Python is a high level programming language well suited for scientific
applications. It is commonly used for scientific computing, because it
features a rich library of functions and applications suited for this
and because it is a very intuitive and easy language to
learn. Furthermore, it is written in C and can easily be extended with
user designed modules. Python features easy usage and incorporation of
Fortran applications as well, which is historical noted for being the
foremost language used for scientific computing.

The scientific module for Python, SciPy, introduces a large collection
of scientific tools with applications in a wide range of relevant
areas - statistical analysis, engineering, linear arithmetics etc.
SciPy depends on another module NumPy, which introduces a fast and
convenient multidimensional array.

The NumPy multidimensional array, ndarray (also referred to as the
NumPy array), features a vast library of methods for various
computations and reshaping. These methods include basic linear
algebraic functions, basic statistics, and advanced array
manipulations. It relies on external libraries and NumPy core
functionalities.

Unfortunately it is currently impossible to utilize the powers of
SciPy and NumPy on the \CBE{}, because Python is not build to utilize
the specific architecture of the \CBE{}. \\

This master thesis will implement a version of NumPy, that will
utilize the \CBE{}.

We will study the possibilities of how to organize data on the \CBE\
to efficiently run scientific calculations, transfer the data back and
forth between the elements in the \CBE{} and incorporate it in NumPy
as its multidimensional array. for this purpose a data structure will
be developed, that will suit all demands and requirements as best as
possible.

Furthermore, a fully functional framework for running NumPy code
fragments efficiently on the \SPEv{'s} will be developed through
analysis of different possibilities.

This will lead to a conceptual version of NumPy that will utilize the
powers and potential of the \CBE{}. Furthermore, a series of programs
will be implemented to show the efficiency of the chosen data
structure and the implementation. These programs will serve as proof
of the concept for utilizing the \CBE\ through Python.\\

To test the data structure efficiently, some computational demanding
programs will be implemented. Such as solving a set of linear
equations and Monte Carlo $\pi$.

A non \CBE\ optimized version of NumPy will serve as basis of
comparison. It will of course only be using the Power PC part of
the \CBE{}. To keep the perspective as real as possible,
the \CBE\ enhanced NumPy version will also be compared to a version
running on an Intel quad core and dual core, see
table \ref{tbl:setup}. The three versions will be compared as ``out of
the box'' installations. This means, that a standard installation of
NumPy will be used.

\begin{table}
\begin{center}
\begin{tabular}{|l|c|c|}
\hline
                & Quad Core & Dual Core \\
\hline
Clock Speed     & 2.4GHz    & 1.66GHz   \\
Cache           & 8MB L2    & 2 MB L2   \\
FSB             & 1066MHz   & 667MHz    \\
Memory          & 2GB       & 4GB       \\
Mem Clock speed & 800MHz    & 533MHz    \\
\hline
\end{tabular}
\caption{Specifications of the two PC's used for comparisons.\label{tbl:setup}}
\end{center}
\end{table}

Our implementation will primarily be targeted at a single PlayStation
3, since it was what we had at our disposal. The key difference
between a PlayStation 3 and a common Cell processor, is the lack of
two compute intensive processors. So, instead of having eight processors, only
six are available(not counting the main Power PC processor).

%Of course we will analyze and discuss the circumstances where a
%``real'' Cell processor would run our implementation. Furthermore, we
%will discuss the situation with a Blade Center\footnote{A dual Cell
%processor, consisting of 16 compute intensive processor and two
%Power PCs.}.\\

Since NumPy is very large and consists of many methods and functions,
it is beyond the scope of this thesis to implement them all. The
thesis will instead involve specific areas of NumPy that has great
relevance with regard to scientific computing.


\section{Motivation}

Given an efficient \CBE\ version of NumPy, SciPy should in overall be
able to perform good. This could lead to a migration from the
commercial mathematical application Matlab, to an open source \CBE\
enhanced version of SciPy/NumPy. With SciPys large scientific library
and NumPy's multidimensional arrays, the combination is close to match
Matlab, user interface wise and syntactical.

Furthermore, general application programmers using NumPy, will, free
of charge, be able to harness the powers of the \CBE{}.\\

As a basis of this masters thesis, we will build on some of the
conclusions given by Rasmus T{\o}nnesens masters
thesis\cite{scipy}. In his thesis he concludes, that when implementing
scientific programs, using Python, on the \CBE{}, it is an advantage
to use a more sophisticated data structure for data.

He uses \FHB\ as his general data layout, because it utilizes the
hierarchic memory layout on the \CBE{} very efficiently, he receives
great performance. It will therefore serve as a starting point in our
thesis. The data structure will be analyzed and implemented into
NumPy. We will see how well it cooperates with all the other elements
in NumPy and how it performs.


\section{Thesis Outline}

In this master thesis we will have the most analytical chapters
first. This will be \REFCHAP{chapter:cell}, where the \CBE\
architecture is analyzed and examined, \REFCHAP{chapter:python}, where
the Python structures and NumPy structures are analyzed and discussed,
and
\REFCHAP{chapter:analysis}, where the underlying data structure for our implementation is analyzed, with relation to the previous chapter. 

In \REFCHAP{chapter:dispatcher} we will discuss the findings from the
first chapters and form a way and method of programming
the \SPUv{'s}. \REFCHAPP{chapter:numpy} introduces the insides of
NumPy, how to incorporate new functionality, by examples and
experiences.

In \REFCHAP{chapter:implementation}, we
discuss how the actual system is implemented on the \CBE{}.

This all adds up to \REFCHAP{chapter:testprograms}, where the final
implementation of NumPy on the \CBE{}, will be tested, given three
distinct test programs. These three tests will be benchmarked
in \REFCHAP{chapter:performance}.

%\REFCHAPP{chapter:tools} will briefly demonstrate the developed tools for easing the project process.

Finally, \REFCHAP{chapter:conclusion}, will conclude this master
thesis, sum up the derived conclusions achieved during the overall
process and testing.

This master thesis is primarily directed for people interested in
Python, Cell BE and scientific computing. It is assumed, that the
reader is familiar with the programming languages Python and C, and
has a basic understanding of computer science.

Our implementation can be found on the web site
numpycbe.ejbyurterne.dk or the web site numpycbe.gymnerds.dk.



% , henceforth called NumPyCBE
