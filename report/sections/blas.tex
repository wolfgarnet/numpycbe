This section will contain a description and an analysis of the
\BLAS\ library. 

The \BLAS\ library contains several linear algebra building blocks
that can be used for more complex operations. This is the case with
\LAPACK{}, which is a set of numerical linear algebra subroutines, that
is used for solving commonly occuring problems.

The \BLAS\ library contains three levels of linear algebra
subroutines. The first level for \BLAS\ was proposed in 1973 by
C. L. Lawson et al\cite{proposal_blas}. In 1985 J. J. Dongarra et al
proposed an extended \BLAS\ as a second
level\cite{proposal_extended_blas}. 

Both level one and level two \BLAS\ are not very well suited for
architectures with a hierarchy of memory and parallel computing. For
those types of architectures, it is often preferable to partion the
matrices into blocks and then compute matrix matrix operations on
block level\cite{blas_level_3}. This strategy gives the opportunity to
reuse the data while it is in memory and less movement of the data is
done.

\begin{enumerate}
\item{Vector vector operations.}
\item{Vector matrix operations.}
\item{Matrix matrix operations.}
\end{enumerate}

In this chapter each of the levels will be discussed and analyzed in
details to find any quirks and obstacles so that the implementation
will be eased on the \CBE{}. Finally, a section describing how the
best implementation strategy on the \CBE\ is presented. \\

The \BLAS\ routines are build up of a name and a prefix. The prefix
indicates which types of arguments and return values the routine uses
and produces. An overview of the prefixes can be seen in
table \ref{tbl:blaspre_overview}.

\begin{table}
\begin{tabular}{|l|l|c|p{8cm}|}
\hline
\bf{Type} & \bf{Precision} & \bf{Prefix} & \bf{Note} \\
\hline
\hline
Real    & Single & S   & \\
\hline
Real    & Double & D   & \\
\hline
Real    & N/A    & SDS & Single precision input, calculated as double, returned as single.\\
\hline
Complex & Single & C   & \\
\hline
Complex & N/A    & SC  & Single precision complex as input, single precision real as output. \\
\hline
Complex & N/A    & CS  & Single precision complex scaled by single precision real. \\
\hline
Complex & Double & Z   & \\
\hline
Complex & N/A    & DZ  & Double precision complex as input, double precision real as output. \\
\hline
Complex & N/A    & ZD  & double precision complex scaled by double precision real. \\
\hline
\end{tabular}
\caption{Overview of prefixes in BLAS routines.\label{tbl:blaspre_overview}}
\end{table}

Not all routines can be used with every prefixes, of course the
conjugated dot product can only be used with complex numbers and
likewise, the Givens rotations can only be aplied to real numbers.

\section{Level 1 BLAS}

The first level of \BLAS\ consists of vector operations and requiring
$O(n)$ scalar operations, where $n$ is the size of the vector
involved. The required level one routine proto types can be seen in
listings \ref{lst:cblaslevelone}. (Move it to appendix???)

\BLAS\ level one routines contains rotations, swap, scaling, copy,
addition of a scaled vector, dot products, euclidean norms, sum and
max. The more interesting of the routines will be analyzed to refine
the implementation for the \CBE{}.

\examplecodesmall{code/blaslevel1.c}{Level one BLAS}{lst:cblaslevelone}




This level only operates on one or two vectors and can return only one
vector or a single scalar. The complexity of level one routines is
low, this can lead to an excessive use of the bandwith, thus this
level is bandwith bound.

A benefit of the simplicity of this level is, that there are no
interdependencies between the vector elements and, as it will be seen
with level three, the results have there own output. This should lead
to an even simpler code with almost no logic for storing or getting
data.

Some of the routines are simplified to minimize the computational
overhead, thus instead of computing the $L^2$-norm, as
the \function{xNRM2} does, the less computational $L^1$-norm is used
instead. The complex sum routines, \function{SCASUM}
and \function{DZASUM} computes a modified $L^1$-norm,
equation \ref{eq:sum_simp}, whereas the classical $L^1$-norm would be
as in equation \ref{eq:sum_org}\cite{blas_for_fortran}.

\begin{equation}
w = \sum_{i=1}^n\{[Re(x_i)]^2 + [Im(x_i)]^2\}^{\frac{1}{2}}
\label{eq:sum_org}
\end{equation}

\begin{equation}
w = \sum_{i=1}^n\{|Re(x_i)| + |Im(x_i)|\}
\label{eq:sum_simp}
\end{equation}.

Similarily, routines \function{ICAMAX} and \function{IZAMAX} are
simplified to endorse a less expensive implementation of finding the
index for the maximum element in a vector. See
equation \ref{eq:max_org} for the classical version and
equation \ref{eq:max_simp} for the simplified.

\begin{equation}
[Re(x_i)]^2 + [Im(x_i)]^2 = \max_{i=1}^n\{[Re(x_i)]^2 + [Im(x_i)]^2\}
\label{eq:max_org}
\end{equation}

\begin{equation}
[Re(x_i)] + [Im(x_i)] = \max_{i=1}^n\{|Re(x_i)| + |Im(x_i)|\}
\label{eq:max_simp}
\end{equation}

The simpliefied versions of \function{SCASUM} and \function{DZASUM},
denoted $||x||_{\tilde{i}}$, and the simpliefied versions
of \function{ICAMAX} and \function{IZAMAX}, denoted
$||x||_{\tilde{\infty}}$, are approximations of the classical norms
$||x||_{i}$ and $||x||_{\infty}$. That is, $||x||_{i} \le
||x||_{\tilde{i}} \le 2^{\frac{1}{2}}$ and $||x||_{\infty} \le
||x||_{\tilde{\infty}} \le 2^{\frac{1}{2}}$.

% Jeg skal lige finde ud af hvorfor det kan lade sig gøre!!!

\subsection{The Dot Products}

The \BLAS\ level one prescribes three types of dot products.

\begin{itemize}
\item{Regular dot product, for real numbers}
\item{Conjugated dot product, for complex numbers}
\item{Uncunjugated dot product, for complex numbers}
\end{itemize}

The regular dot product is only used for vectors of real numbers and
is of the form

\[v \cdot w = \sum_{i=1}^n = v_i w_i = v_1 w_1 + v_2 w_2 + \cdots + v_n w_n \].

The conjugated dot product is only for complex numbers and is defined as

\[v \cdot w = \sum_{i=1}^n = \overline{w_i} v_i \],

where $\overline{w_i}$ is the complex conjugate. The complex conjugate
of a complex number is defined as changing the sign of the imaginary
part, thus $z = a + bi$ is $z = a - bi$. The unconjugated version
retains the original sign of the imaginary part.

The regular dot product can be called with single precision,
calculated using double precision arithmetics and return a single
precision result.

Vectors can be thought of as $1 \times n$ matrices for row vectors and
$m \times 1$ matrices for column vectors. Thus treating the dot
product as a matrix multiplication. Therefore the standard notation
$v \cdot w$ is inadequate, due to the nature of matrix
multiplication. Either of the vectors must be transponed and thus
giving following equation, $v^T w$ and $v^H w$ for the complex
conjugate dot product.

\subsection{Givens Rotations}

\BLAS\ level one also prescribes two types of rotations, Givens
rotations and modified Givens rotations, \function{xROTG}
and \function{xROTMG} respectively. The two rotation routines
calculates the elements of a Givens rotation matrix.

The rotations can be used to zero out indivdual elements in a matrix
or a vector, which can be adopted in other routines. Examples???

A Givens rotation matrix is an identity matrix with for elements
substituted. These four elements are $M_{ii} = c = cos(\theta)$,
$M_{kk} = c = cos(\theta)$, $M_{ik} = s = sin(\theta)$ and $M_{ki} =
-s$. The matrix looks like the following.

\[\left[
\begin{array}{ccccccc}
1      & \cdots & 0      & \cdots & 0      & \cdots & 0 \\
\vdots & \ddots & \vdots &        & \vdots &        & \vdots \\
0      & \cdots & c      & \cdots & s      & \cdots & 0 \\
\vdots &        & \vdots & \ddots & \vdots &        & \vdots \\
0      & \cdots & -s     & \cdots & c      & \cdots & 0 \\
\vdots &        & \vdots &        & \vdots & \ddots & \vdots \\
0      & \cdots & 0      & \cdots & 0      & \cdots & 1 \\
\end{array}
\right]\]

What the routines, \function{xROTG} and \function{xROTMG} does, is
calculating the two variables, \variable{c} and \variable{s}.

The routine \function{xROTG} takes four 

% meget mere, men lige nu gider jeg ikke...

\subsection{Findings/Conclusion}

The routines of \BLAS\ level one are very atomic. That is, nearly
every(must be checked) routine can be implemented using very little
code. For large data sets, caution must be taken due to the number of
calculation on each data element. That is, the transfer / workload
ratio problem from section \ref{sec:transferworkload}.


\begin{tabular*}{0.95\textwidth}{|l|l|l@{\extracolsep{\fill}}|}
\hline
\bf{Operation} & \bf{Supported precisions} & \bf{Comment} \\
\hline
ASUM & Single, double, single complex, double complex & \\
\hline
AXPY & Single, double, single complex, double complex & \\
\hline
COPY & Single, double, single complex, double complex & \\
\hline
DOT & Single, double\footnotemark & \\
\hline
DOTU & Single complex, double complex & \\
\hline
DOTC & Single complex, double complex & \\
\hline
NRM2 & Single, double, single complex, double complex\footnotemark & \\
\hline
ROTG & Single, double & \\
\hline
ROTM & Single, double & \\
\hline
ROTMG & Single, double & \\
\hline
SCAL & Single, double, single complex, double complex\footnotemark & \\
\hline
SWAP & Single, double, single complex, double complex & \\
\hline
\end{tabular*}

\addtocounter{footnote}{-3}
\stepcounter{footnote}\footnotetext{Complex vectors can be scaled with floating points of the same precision.}
\stepcounter{footnote}\footnotetext{Complex vectors will return floating points of the same precision.}
\stepcounter{footnote}\footnotetext{Single precision can be calculated with double precision and can be returned with either single or double precision.}

\section{Level 2 BLAS}

The first level of \BLAS\ consists of matrix vector operations and
requiring $O(m n)$ scalar operations, where $m$ and $n$ are the
dimensions of the matrix involved. The level two routines could be
implemented by a series of calls to level one \BLAS\ routines. It is
not recommended though, because the level two \BLAS\ is performing
basic operations at one level higher than level
one \BLAS{}\cite{extended_blas_for_fortran}.


The required level two routine proto types can be seen in
listings \ref{lst:cblasleveltwo}.

\examplecodesmall{code/blaslevel2.c}{Level two BLAS}{lst:cblasleveltwo}

\subsection{Matrix Vector Products}

The level two \BLAS\ specifies N matrix vector products. One for a
general matrix, where the matrix operated on is either the normal
matrix, its transpose or its conjugate transpose(for complex
matrices).

That is,

\[
y \gets \alpha A x + \beta y, \quad
y \gets \alpha A^T x + \beta y \quad \textrm{and}  \quad
y \gets \alpha \overline{A^T} x + \beta y 
\]

respectively.

\subsubsection{Band Matrices}

One for band matrices of the form

\[\left[
\begin{array}{ccccccc}
M_{11}  & M_{12} & 0      & \cdots & 0         & \cdots    & 0 \\
M_{21}  & M_{22} & M_{23}  & \ddots  &         &           & \vdots \\
0      & M_{32}  & M_{33} & \ddots &           &           & 0 \\
\vdots &        & \ddots & \ddots & \ddots    &           & \vdots \\
0      &        &        & \ddots & M_{m-2n-2} & M_{m-2n-1} & 0 \\
\vdots &        &        &        & M_{m-1n-2} & M_{m-1n-1} & M_{m-1n} \\
0      & \cdots & 0      & \cdots & 0         & M_{mn-1}   & M_{mn} \\
\end{array}
\right]\]

where the super diagonal, $KU$, and sub diagonal, $KL$ is 1. In this
case the matrix is a swuare matrix, it doesn't have to be. The super
and sub diagonal refers to the width of the diagonal band, the upper
and lower part of the matrix respectively, where the non zero elements
of the matrix are confined. The width of the band is $KU + KL +
1$. Generally, the zero components of the matrix can be written as in
equation \ref{eq:bandmatrix}.

\begin{equation}
m_{ij} = \left\{
\begin{array}{rl}
0 & \mathbf{if} \quad j < i - KL \\
0 & \mathbf{if} \quad j > i + KU \\
\end{array} \right.
\label{eq:bandmatrix}
\end{equation}

A diagonal matrix is a matrix with $KU + KL = 1$.

As for the general matrix, the band matrix can also be operated as a
normal matrix, transposed matrix and a conjugated transposed matrix.

\subsubsection{Hermitian Matrices}

A hermitian matrix is a square matrix, $M = m_{ij}$, where $M = M^H$
and $M^H$ is the transposed matrix conjugate, or
$\overline{M}^T$. This is equivalent to $m_{ij}
= \overline{m_{ji}}$. As a result of this, the diagonal of the matrix
are real numbers and the other numbers may be complex. In \BLAS\ only
complex numbers are allowed, due the homogeneous nature of the library
and therefore, the diagonal numbers are complex with there imaginary
part as zero.

The following two matrices are both hermitian, $\mathbf{NH}$ is normal
hermitian matrix and $\mathbf{BH}$ is a \BLAS\ specific hermitian
matrix. As a consequence of this, the \BLAS\ hermitian matrix can be
stored as a symmetric matrix in either the upper or lower triangle,
because the opposite triangle is its complex conjugate.

\[
\mathbf{NH} = \left[
\begin{array}{ccc}
1       & -2 + 3i & 2  \\
-2 - 3i & 32      & i  \\
2       & -i      & -1 \\
\end{array}
\right]
\qquad
\mathbf{BH} = \left[
\begin{array}{ccc}
1 + 0i  & -2 + 3i & 2 + 0i      \\
-2 - 3i & 32 + 0i & i       \\
2 + 0i       & -i      & -1 + 0i \\
\end{array}
\right]
\]

The hermitian matrices can also have the band feature as the general
matrices have. This only requires a symmetric bandwidth of $K$, for
which the values are held. Everything else is zero, $z = 0 + 0i$.

\subsubsection{Symmetric Matrices}

Symmetric matrices have almost the same property as a hermitian
matrix. Though the symmetric matrices only operates on real numbers,
the transposed conjugate makes no sense and only

\[A = A^T\]

is required. Symmetric matrices can also be symmetrically banded. 


\subsubsection{Triangular Matrices}

A triangular matrix is a square matrix, where either the triangle
above or below the main diagonal is zeroed, as shown 

\[
\mathbf{UM} = \left[
\begin{array}{ccccc}
M_{11}  & M_{12} & M_{13} &  \cdots  & M_{1m} \\
0       & M_{22} & M_{23}  & \cdots  & M_{2m} \\
\vdots  &        & \ddots &  \ddots &  \vdots \\
\vdots  &        &        &  M_{m-1m-1} & M_{m-1m} \\
0       & \cdots & \cdots  & 0      &  M_{mm} \\
\end{array}
\right]
\qquad
\mathbf{LM} = \left[
\begin{array}{ccccc}
M_{11}  & 0     & \cdots &   \cdots  & 0 \\
M_{21}  & M_{22} &        &          &\vdots \\
M_{31}  & M_{32} & \ddots &          &  \vdots \\
\vdots & \vdots & \ddots &  M_{m-1m-1} & 0 \\
M_{m1}  & M_{m2} & \cdots  & M_{mm-1} &  M_{mm} \\
\end{array}
\right]
\]

where $\mathbf{UM}$ is an upper triangular matrix and $\mathbf{LM}$ is
a lower triangular matrix. The triangular matrix can either be unit or
not, that is, whether the main diagonal of the matrix is ones.

As for general matrices the triangular matrix can also be computed as
its own transposed and, for complex triangular matrices, its own
transposed conjugate.

The triangular matrix can also be banded with $k+1$ diagonals.

\subsubsection{Packed Matrices}

Symmetrically matrices, both symmetric and hermitian, and triangular
matrices can be packed. That is, the data can be stored in less space
than in a regular matrix. Because both the upper or lower triangle
part of the matrix, store the same numbers(or, in the hermitian case,
its complex conjugate).

Using packed matrices saves space, but to access the elements more
advanced logic is required. This logic can be complex and the trade
off is not always beneficial.

\subsection{Triangular Equation Solvers}

Solving a matrix equations of the form $\mathbf{M} x = y$, the matrix
equation can be written as a system of linear equations

\[
\left[
\begin{array}{ccccccccc}
M_{11} x_{1} & + & M_{21} x_{2}  & + & \cdots  & + & M_{1n} x_{1} & = & y_{1}\\
M_{21} x_{1} & + & M_{22} x_{2}  & + & \cdots  & + & M_{2n} x_{2} & = & y_{2}\\
\vdots      &   & \vdots       &   &         &   & \vdots      &   & \vdots\\
M_{m1} x_{1} & + & M_{m2} x_{2}  & + & \cdots  & + & M_{mn} x_{n} &  = & y_{m}\\
\end{array}
\right]
\]

where $\mathbf{M}$ is a general matrix, $x$ is the unknown and $y$ the
result vector, the procedure can be very complex. On the other hand,
if the matrix is a triangular matrix, solving the equation is
easy. Using a process called ``forward substitution'' for lower
triangular matrices or using a process called ``back substitution''
for upper triangular matrices, the solution is easily found. The
process is iterative. The first equation, with only one unknown, is
solved, $M_{11} x_1 = y_1$, for lower triangular matrices. The result
is then substituted forward to the next equation, thus leaving only
one unknown. The system of linear equations for a lower triangular
matrix is as shown

\[
\left[
\begin{array}{ccccccccc}
M_{11} x_{1} &   &        & &        & & &  = & y_{1}\\
M_{21} x_{1} & + & M_{22} x_{2}  & &        & & &  = & y_{2}\\
\vdots &   & \vdots & & \ddots & & &    & \vdots\\
M_{m1} x_{1} & + & M_{m2} x_{2}  & + & \cdots & + & M_{mn} x_{n} &  = & y_{m}\\
\end{array}
\right]
\]

The process of back substitution is analogous, just performing the
steps backwards.

\subsection{Matrix Rank Updates}

Rank one updates for a general matrix looks like

\[
\mathbf{M} = \alpha x y^T + \mathbf{M}
\]

where $\mathbf{M}$ is an $m \times n$ matrix of either real numbers or
complex numbers. $x$ is an $m$ element vector and $y$ is an $n$
element vector, both are of the same type as $\mathbf{M}$. $\alpha$ is
a scalar and is also of the same type as $\mathbf{M}$.

The conjugate of $y$ can also be used, thus only complex numbers can
be used

\[
\mathbf{M} = \alpha x y^H + \mathbf{M}
\]

Rank one updates can also be applied to hermitian matrices, packed and
not packed. The equation looks like

\[
\mathbf{H} = \alpha x x^H + \mathbf{H}
\]

where $\mathbf{H}$ is a hermitian $n \times n$ matrix, $\alpha$ is a
complex scalar and $x$ is a complex $n$ element vector. For hermitian
matrices, rank two updates are possible as well. They are of the form

\[
\mathbf{H} = \alpha x y^H + \overline{\alpha} y x^H + \mathbf{H}
\]

\noindent 
Analogous to hermitian matrices, symmetric matrices can also be
applied with both rank one and rank two updates. (Mere om de symmetriske???)

\subsection{Findings/Conclusion}

As it was seen with level one routines, level two also ``suffers''
from simplicity.

\section{Level 3 BLAS}

The third level of \BLAS\ is matrix matrix operations of the order
$O(n^3)$.

\subsection{Matrix Multiplication}

The general matrix multiplication in \BLAS\ level three is of the form

\[
\mathbf{C} \gets \alpha \mathbf{A} \mathbf{B} + \beta \mathbf{C}
\]

So, actually it is a matrix matrix multiply and add operation. The
general matrix multiplication can be used with both real numbers and
complex numbers, transposed and the conjugate transposed. The
following table shows the types of matrix multiplication

\begin{tabular}{|c|ccc|}
\hline
            & \bf{Normal A} & \bf{Transposed A} & \bf{Conj. transposed A} \\
\hline
\bf{Normal B} & $\mathbf{C} \gets \alpha \mathbf{A} \mathbf{B} + \beta \mathbf{C}$  
              & $\mathbf{C} \gets \alpha \mathbf{A}^T \mathbf{B} + \beta \mathbf{C}$ 
              & $\mathbf{C} \gets \alpha \mathbf{A}^H \mathbf{B} + \beta \mathbf{C}$\\
              & A is $m \times k$, B is $k \times n$                                
              & A is $k \times m$, B is $k \times n$                                 
              & A is $k \times m$, B is $k \times n$ \\

\hline
\bf{Transposed B} & $\mathbf{C} \gets \alpha \mathbf{A} \mathbf{B} + \beta \mathbf{C}$  
                  & $\mathbf{C} \gets \alpha \mathbf{A}^T \mathbf{B}^T + \beta \mathbf{C}$ 
                  & $\mathbf{C} \gets \alpha \mathbf{A}^H \mathbf{B}^H + \beta \mathbf{C}$\\
                  & A is $m \times k$, B is $k \times n$                                
                  & A is $k \times m$, B is $n \times k$                                 
                  & A is $k \times m$, B is $n \times k$ \\

\hline
\bf{Conj. transposed B} & $\mathbf{C} \gets \alpha \mathbf{A} \mathbf{B}^H + \beta \mathbf{C}$  
                        & $\mathbf{C} \gets \alpha \mathbf{A}^T \mathbf{B}^H + \beta \mathbf{C}$ 
                        & $\mathbf{C} \gets \alpha \mathbf{A}^H \mathbf{B}^H + \beta \mathbf{C}$\\
                        & A is $m \times k$, B is $n \times k$                                
                        & A is $k \times m$, B is $n \times k$                                 
                        & A is $k \times m$, B is $n \times k$ \\

\hline
\end{tabular}

where $\mathbf{C}$ is always $m \times n$.

\section{BLAS on the CELL BE}

When implementing \BLAS\ on the \CBE{}, some care must be taken. First
of all the data must be transferred to the \CBE{}. What is the optimal
data structure and how is it best organized relative to the \DMA\
transfers. Some of the routines alters the data, so there are
interdependencies among the data.

Level one and two \BLAS\ is best suited with a contiguous memory
layout, where level three is best suited with blocks of data.
