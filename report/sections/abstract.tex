NumPy is a well founded module for Python for computing scientific
calculations. Its most important features are its multidimensional
array and its functionality to operate on it. NumPy is currently
supported by major platforms and architectures, but its core
functionality is only implemented sequentially. Even though NumPy runs
on the Cell Broadband Engine(Cell BE), it does not utilize the Cell
BE's compute intensive processors.

By extending NumPy to utilize the extra processors, it can potentially
be 20 to 25 times faster than its main processor, in some cases even
faster. Since NumPy is a Python module, it is very easy to use and
program, because it has an intuitive syntax. By developing such a
module, non-computer scientist can use the Cell BE without having to
get their hands dirty.

In this master thesis we will implement a version of NumPy, that
utilizes the specific hardware of the Cell BE. It will be based on a
fast and flexible kernel for dispatching NumPy functionality on the
compute intensive processors. This kernel will also be in charge of
allocating memory according to the given needs.

To test the implementation, we have developed three test programs,
which will use different NumPy functionalities, both core
functionality, but also functionalities given from external libraries,
such as linear algebraic routines from BLAS and LAPACK.

We will compare our implementation, running on a PlayStation 3, to
NumPy running on different architectures.
