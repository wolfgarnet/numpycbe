\chapter{Guides}

\section{Dispatcher Guide}

This is a brief guide of how the Dispatcher works.

\subsection{Start Up}

The dispatcher must be loaded into the \SPE{}'s \LS{}. This could be
done with:

\examplecode{code/dispatcherload.c}{Loads an SPE ELF into LS}{lst:dispatcherload}

Each \SPE\ must be assigned a unique identification, starting at zero
and counting upwards. Furthermore, each \SPE\ must be given a seed to
initialize its random generator. This could be done by retrieving the
number of seconds by the function \function{time()}.

Given some defines in the file \function{common\_spu.h}, the sizes of the
memory layout in the \LS\ can be changed. Their default values gives
16KB memory per data block, 88KB for the shader and 12KB per actual
shader.

\examplecode{code/ls_layout.c}{The default values for the LS layout}{lst:lslayout}

\subsection{Operations}

Operations for the dispatcher, is given by the operation structure
\struct{Operation\_t}, see code listing \ref{lst:operation_t}.

\examplecode{code/operation_t.c}{The Operation\_t structure}{lst:operation_t}

\subsection{States}

The dispatcher, when successfully initialized, becomes a state
machine. It implements several types of states, whereas they can be
divided into two subgroups:

\begin{itemize}
\item{Functionality with an actual return values}
\item{Functionality with no return value}
\end{itemize}

Functionality with no returns values, do indeed return something, just
not something that involves the dispatcher, but only the given
shader. An actual return value could be something the \PPE\ should
complete. For example \function{sum}, each \SPE\ sums a specific part
of the array and returns their intermediate results to the \PPE{},
whom combines the results into one.

The main difference is, that the \PPE\ program informs each of the
dispatchers where their private intermediate result should be put in
main memory. Otherwise, they look alike.

In table \ref{tbl:states}, the various states can be seen. 

\begin{table}
\begin{tabular}{|c|c|c|c|c|c|}
\hline
State         & Operation ID  & Upload shader & Run shader & Specific return value & Signals \\
\hline
$0 \to 99$    & $0 \to 99$    & Yes           & Yes       & No                     & Yes \\
$100 \to 199$ & $0 \to 99$    & Yes           & Yes       & Yes                    & Yes \\
$200 \to 299$ & $0 \to 99$    & Yes           & No        & No                     & Yes \\
$300 \to 399$ & $0 \to 99$    & No            & Yes       & No                     & No  \\
$400 \to 499$ & $0 \to 99$    & Yes           & No        & Yes                    & Yes \\
$500 \to 599$ & $0 \to 99$    & No            & Yes       & Yes                    & No  \\
\hline
$1000$        & $\varnothing$ & No            & No        & No                     & Yes \\
\hline
\end{tabular}
\caption{Types of states for dispatcher\label{tbl:states}}
\end{table}

\section{Shader Guide}

\section{NumPyCBE}

\subsection{Installation}

NumPyCBE is installed as easily as NumPy, though currently the
installation must be done in \textit{/home/user/numpycbe}. And the
actual install command is thus, in that specific directory,
\texttt{``python setup.py install --prefix=.''}.  If the special
NumPyCBE 64 block optimized version is wanted, the command
\texttt{``python setup64.py install --prefix=.''}, should be used
instead.

To ensure that NumPyCBE can be loaded into any python script
everywhere, the following addition to the \file{.bashrc} file
required:

\examplecode{code/bashrc}{.bashrc file}{lst:bashrc}


\subsection{Functions}

The functionality implemented in NumPyCBE is identical to
NumPy. However, some of the syntax is a little different. This applies
to general operator overloading. For example the index operator and
plus operator.

Table \ref{tbl:overload}, shows the corresponding NumPyCBE syntax.

\begin{table}
\begin{tabular}{|c|c|c|c|c|}
\hline
Side  & Function   & NumPy Syntax & NumPyCBE syntax   & Notes \\
\hline
%Left & Assignment & var[x]       & setIndex(a,x,r,c) & a is the array, x is the value and r\newline and c are the row and column in the array \\
Left  & Assignment & var[i] = x   & setIndex(a,x,r,c) & \mpage{5}{a is the array, x is the value and r and c are the row and column in the array} \\
\hline
Left  & Assignment & var[i] = x   & SetIndex(a,x,i,j) & \mpage{5}{a is the array, x is the value and i and j are the column and row in the array} \\
\hline
Right & Indexing   & x = var[i]   & index(a,x,r,c)    & \mpage{5}{a is the array, x is the value and r and c are the row and column in the array} \\
\hline
Right & Indexing   & x = var[i]   & Index(a,x,i,j)    & \mpage{5}{a is the array, x is the value and i and j are the column and row in the array} \\
\hline
Right & Infix      & a op b       & op(a,b)          & \mpage{5}{a and b are ndarrays or one of them is a scalar, and op is for example +, /, or +=} \\
\hline
\end{tabular}
\caption{Operator overloading conversion\label{tbl:overload}}
\end{table}

In table \ref{tbl:numpycbefuncs} specific NumPyCBE functionalities are listed.


\begin{table}
\begin{tabular}{|c|c|}
\hline
Function        & Notes \\
\hline
SetBlockSize(x) & Sets the currently used block size for FHB\\
\hline
Print(a)        & Prints the ndarray\\
\hline
PrintS(a)       & Prints the ndarray with scientific notation\\
\hline
Create(x,y,v)   & Creates an ndarray with dimensions $y \times x$, with default value v\\
\hline
\end{tabular}
\caption{NumPyCBE specific functionality\label{tbl:numpycbefuncs}}
\end{table}

The following lists the implemented NumPy functionality,

\begin{itemize}
\item{add}
\item{arange}
\item{array}
\item{divide}
\item{len}
\item{lessequal}
\item{multiply}
\item{random}
\item{shape}
\item{subtract}
\item{sum}
\item{solve}
\item{zeros}
\end{itemize}

Though it should be noted, that not all arguments are valid. For
example, only multiply supports slicing.

%\chapter{Test Programs}
%\prepchapter
%\label{chapter:code}

%\section{Monte Carlo Python Code}
%\label{sec:mc_code}

%\subsection{NumPy}
%\examplecode{../code/numpycbe/testprograms/MonteCarlo_org.py}{setup.py file for NumPyCBE}{lst:setup}

%\subsection{NumPyCBE}

%\section{SOR Python Code}
%\label{sec:sor_code}

%\subsection{NumPy}
%\subsection{NumPyCBE}

%\section{Solve Python Code}
%\label{sec:solve_code}

%\subsection{NumPy}
%\subsection{NumPyCBE}









\chapter{Code Listings}
\prepchapter
\label{app:listings}

\section{Shader Code}

\subsection{Generic Binary Function Framework}
\label{app:gbff}
\examplecode{code/genericbinaryshader.c}{Framework for the generic binary shader functionality}{lst:genericshader}


\subsection{Generic Binary Addition}
\label{app:g-addition}
\examplecode{code/genericadd.c}{The \function{\_compute} function for array addition}{lst:genericadd}









\chapter{Output}
\prepchapter
\section{Cell Simulator}

\subsection{STRSM}
\label{app:strsm}

\scriptsize
\begin{verbatim}
SPU DD3.0
***
Total Cycle count               780925257
Total Instruction count         1717159134
Total CPI                       0.45
***
Performance Cycle count         773376904
Performance Instruction count   107390610 (103513232)
Performance CPI                 7.20 (7.47)

Branch instructions             26385717
Branch taken                    26336887
Branch not taken                48830

Hint instructions               59175
Pipeline flushes                25628344
SP operations (MADDs=2)         91394472
DP operations (MADDs=2)         0

Contention at LS between Load/Store and Prefetch 78004

Single cycle                                          97227408 ( 12.6%)
Dual cycle                                             3142912 (  0.4%)
Nop cycle                                              2255568 (  0.3%)
Stall due to branch miss                             435706971 ( 56.3%)
Stall due to prefetch miss                                   0 (  0.0%)
Stall due to dependency                              234491926 ( 30.3%)
Stall due to fp resource conflict                            0 (  0.0%)
Stall due to waiting for hint target                     24478 (  0.0%)
Issue stalls due to pipe hazards                             0 (  0.0%)
Channel stall cycle                                     527641 (  0.1%)
SPU Initialization cycle                                     0 (  0.0%)
-----------------------------------------------------------------------
Total cycle                                          773376904 (100.0%)

Stall cycles due to dependency on each instruction class
 FX2        88885 (  0.0% of all dependency stalls)
 SHUF       37431 (  0.0% of all dependency stalls)
 FX3        73876 (  0.0% of all dependency stalls)
 LS         53587157 ( 22.9% of all dependency stalls)
 BR         0 (  0.0% of all dependency stalls)
 SPR        127843215 ( 54.5% of all dependency stalls)
 LNOP       0 (  0.0% of all dependency stalls)
 NOP        0 (  0.0% of all dependency stalls)
 FXB        0 (  0.0% of all dependency stalls)
 FP6        52837423 ( 22.5% of all dependency stalls)
 FP7        23939 (  0.0% of all dependency stalls)
 FPD        0 (  0.0% of all dependency stalls)

The number of used registers are 89, the used ratio is 69.53

Instruction Class                               Insts Issued     Insts Exec    Exec Cycles    Cycles/Inst
---------------------------------------------   ------------   ------------   ------------   ------------
FX2 (EVEN): Logical and integer arithmetic          29042358      3168514       56405820          17.80
SHUF (ODD): Shuffle, quad rotate/shift, mask         1840959      1677592        6855428           4.09
FX3 (EVEN): Element rotate/shift                      118505        95014         334314           3.52
LS   (ODD): Load/store, hint                        62281693     36581870      246346904           6.73
BR   (ODD): Branch                                  51976694     26385717      156769436           5.94
SPR  (ODD): Channel and SPR moves                  102314174     25606786      383649561          14.98
LNOP (ODD): NOP                                       280979       165982              0           0.00
NOP (EVEN): NOP                                      2200455      2177039              0           0.00
FXB (EVEN): Special byte ops                               0            0              0           0.00
FP6 (EVEN): SP floating point                       11424309     11424309       68545854           6.00
FP7 (EVEN): Integer mult, float conversion            108336       107787         345142           3.20
FPD (EVEN): DP floating point                              0            0              0           0.00

dumped pipeline stats
\end{verbatim}


\section{SPU Timing Tool}

\subsection{STRSM}
\label{app:strsmtime}

\scriptsize
\begin{verbatim}
                                                               _compute:
0D                                       89                    	clgti	$2,$4,63
1D                                       8901                  	shlqbyi	$9,$4,0
0D                                        90                   	ori	$21,$6,0
1D                                        9                    	hbrp	# 1
0d                                         0                   	nop	127
1d                                         -1234               	binz	$2,$lr
0D                                           2345              	shli	$10,$4,6
1D 0123456                                   23456789          	hbrr	.L46,.L43
0                                             34               	andi	$6,$5,63
0                                              4567            	shli	$4,$4,8
0                                               56             	sfi	$23,$9,64
0                                                67            	a	$5,$6,$10
0                                                 78           	ai	$38,$21,16
0  01                                              89          	shli	$3,$5,2
0D 0                                                9          	a	$20,$7,$4
1D                                                  9          	hbrp	# 2
0  01                                                          	ai	$37,$21,32
0   12                                                         	ai	$36,$21,48
0    23                                                        	a	$22,$8,$3
0     34                                                       	ai	$35,$21,64
0      45                                                      	ai	$34,$21,80
0       56                                                     	ai	$33,$21,96
0        67                                                    	ai	$32,$21,112
0         78                                                   	ai	$31,$21,128
0          89                                                  	ai	$30,$21,144
0           90                                                 	ai	$29,$21,160
0            01                                                	ai	$28,$21,176
0             12                                               	ai	$27,$21,192
0              23                                              	ai	$26,$21,208
0               34                                             	ai	$25,$21,224
0                45                                            	ai	$24,$21,240
0                 56                                           	ila	$39,66051
                                                               .L43:
0D                 67                                          	ai	$23,$23,-1
1D                 678901                                      	lqd	$78,0($22)
1                   789012                                     	lqd	$75,0($21)
1                    8                                         	hbrp	# 1
1                     901234                                   	lqd	$76,0($20)
1                      012345                                  	lqd	$73,16($20)
1                       123456                                 	lqd	$70,32($20)
1                        234567                                	lqd	$67,48($20)
0D                        3                                    	nop	127
1D                        3456                                 	rotqby	$77,$78,$22
0D                         45                                  	ai	$22,$22,256
1D                         456789                              	lqd	$64,64($20)
1                           567890                             	lqd	$61,80($20)
1                            678901                            	lqd	$58,96($20)
1                             7890                             	shufb	$40,$77,$77,$39
1                              8                               	hbrp	# 2
1                               901234                         	lqd	$55,112($20)
1                                012345                        	lqd	$52,128($20)
0D                                1                            	nop	127
1D                                123456                       	lqd	$49,144($20)
0D                                 234567                      	fnms	$74,$75,$40,$76
1D                                 234567                      	lqd	$46,160($20)
1                                   345678                     	lqd	$43,176($20)
1                                    456789                    	lqd	$16,192($20)
1                                     567890                   	lqd	$17,208($20)
1                                      678901                  	lqd	$18,224($20)
1                                       789012                 	lqd	$19,240($20)
1                                        890123                	stqd	$74,0($20)
1                                         901234               	lqd	$72,0($38)
0  0                                       -----56789          	fnms	$71,$72,$40,$73
1  -123456                                       ----          	stqd	$71,16($20)
1    234567                                                    	lqd	$69,0($37)
0d    -----890123                                              	fnms	$68,$69,$40,$70
1d         ------456789                                        	stqd	$68,32($20)
1                 567890                                       	lqd	$66,0($36)
0                  -----123456                                 	fnms	$65,$66,$40,$67
1                        -----789012                           	stqd	$65,48($20)
1                              890123                          	lqd	$63,0($35)
0d                              -----456789                    	fnms	$62,$63,$40,$64
1d                                   ------012345              	stqd	$62,64($20)
1                                           123456             	lqd	$60,0($34)
0  012                                       -----789          	fnms	$59,$60,$40,$61
1  ---345678                                       --          	stqd	$59,80($20)
1      456789                                                  	lqd	$57,0($33)
0d      -----012345                                            	fnms	$56,$57,$40,$58
1d           ------678901                                      	stqd	$56,96($20)
1                   789012                                     	lqd	$54,0($32)
0                    -----345678                               	fnms	$53,$54,$40,$55
1                          -----901234                         	stqd	$53,112($20)
1                                012345                        	lqd	$51,0($31)
0d                                -----678901                  	fnms	$50,$51,$40,$52
1d                                     ------234567            	stqd	$50,128($20)
1                                             345678           	lqd	$48,0($30)
0  01234                                       -----9          	fnms	$47,$48,$40,$49
1  -----567890                                                 	stqd	$47,144($20)
1        678901                                                	lqd	$45,0($29)
0d        -----234567                                          	fnms	$44,$45,$40,$46
1d             ------890123                                    	stqd	$44,160($20)
1                     901234                                   	lqd	$42,0($28)
0                      -----567890                             	fnms	$41,$42,$40,$43
1                            -----123456                       	stqd	$41,176($20)
1                                  234567                      	lqd	$15,0($27)
0d                                  -----890123                	fnms	$14,$15,$40,$16
1d                                       ------456789          	stqd	$14,192($20)
1  0                                            56789          	lqd	$13,0($26)
0  -123456                                       ----          	fnms	$12,$13,$40,$17
1    -----789012                                               	stqd	$12,208($20)
1          890123                                              	lqd	$11,0($25)
0d          -----456789                                        	fnms	$9,$11,$40,$18
1d               ------012345                                  	stqd	$9,224($20)
1                       123456                                 	lqd	$8,0($24)
0                        -----789012                           	fnms	$7,$8,$40,$19
0d                             8                               	nop	127
1d                             -----345678                     	stqd	$7,240($20)
0D                                   45                        	ai	$20,$20,256
                                                               .L46:
1D                                   4567                      	brnz	$23,.L43
1                                     5678                     	bi	$lr
\end{verbatim}

